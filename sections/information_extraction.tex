\chapter{Information extraction}
% what is this chapter about

\section{Overview}
The goal of the information extraction component is to obtain useful information from incoming messages. The main tasks performed by this component are named-entity recognition, email extraction, HTML removal, header parsing, quoted text removal and relationship disambiguation. Because this component is the first processing step after entry into the system, it is equipped with methods to deal with potentially dirty data [SECTION]. Once relevant information is extracted, the result is passed as input to the response generation component and is used to generate a convicing, human-like response. For example, the outputs of the named-entity recognition task might be used to compose a personalized greeting in the beginning of the message -- e.g. ``\textit{Hello John}''.

\section{Processing steps}
We apply a series of transformations to the data, in order to extract the relevant 
We apply a series of processing steps to the data, in order to extract the relevant bits of information.

\section{Dealing with dirty data}
Best effort basis. crawler and scrapes, possible not an email at all.
% removing html
% parsing headers
Depending on the origin of the data, we can no longer rely on certain assumptions.

\section{Related entities}
\subsection{Named-entity recognition}
\subsection{Computing pairs}
