\chapter{Introduction}

The document structure should include:
\begin{itemize}
\item
The title page  in the format used above.
\item
An optional acknowledgements page.
\item
The table of contents.
\item
The report text divided into chapters as appropriate.
\item
The bibliography.
\end{itemize}

Commands for generating the title page appear in the skeleton file and
are self explanatory.
The file also includes commands to choose your report type (project
report, thesis or dissertation) and degree.
These will be placed in the appropriate place in the title page. 

The default behaviour of the class is to produce documents typeset in
12 point, 
and appropriate for doubled sided printing
(all new chapters appearing on the first clear right-hand page).
Regardless of the formatting system you use, 
it is recommended that you submit your thesis printed (or copied) 
double sided. 

{\bf NB} please note that the report should be printed single-spaced.
Previously advertised policy of printing in double space has changed
as of November 24th 1999 and is no longer valid.
Space recommendations are revised as follows: 
the dissertation should be around 40 sides in single space printing.
The page limit is 60 sides in single space printing.  Appendices are in
addition to the above and you should place detail here which may be too
much or not strictly necessary when reading the relevant section.

\section{Using Sections}

Divide your chapters in sub-parts as appropriate.

\section{Citations}

Note that citations 
(like \cite{P1} or \cite{P2})
can be generated using {\tt bibtex} or by
creating {\tt thebibliography} environment. This makes sure that the
table of contents includes an entry for the bibliography.
Of course you may use any other method as well.

\section{Class Options}

The only class option available is {\tt parskip}.
It alters the paragraph formatting so that each paragraph is separated by
a vertical space, and there is no indentation at the start of each
paragraph. 
This option is used in the current document.
See {\tt documentclass} in the skeleton file for usage. 

\section{Restrictions}

The class does not allow the use of {\tt listoffigures} or {\tt listoftables}.
