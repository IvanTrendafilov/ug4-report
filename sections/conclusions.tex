\chapter{Conclusions}
This report presented our approach to building a conversational agent for email correspondence with advance fee fraud scammers.
We established the context of the problem and outlined our rationale in Section 1.2 and Chapter 2. In Section 3.1 we provided an
overview of the architecture of the proposed system. In our implementation we encountered a number of interesting challenges.
We summarize here our main contributions.
\begin{description}
\item[Section 4.2] We introduced a set of collector modules which automate the process of obtaining new scam instances and fetching replies to existing conversation threads.
\item[Section 5.2] This section described key information extraction techniques. We detailed our approach to dealing with dirty data, header extraction from non-compliant emails, named-entity recognition, and relation extraction using bigrams and proximity search.
\item[Section 6.2] This section introduced the 419 corpus -- a corpus containing 32,000 advance fee fraund emails in 22 categories.
\item[Section 6.3] MaxEnt. Mention F1 scores.
\item[Section 7.2] bucketing algorithm
\item[Section 7.3] strategies
\item[Section 7.4] response generation

In this section we trained two Maximum Entropy classifiers using data from the 419 corpus. The 

We implement a set of Maximum Entropy classifiers 

We develop techniques for dealing with dirty incoming data and header extraction. In addition, we describe our approach to named-entity recognition and relation extraction using bigrams and proximity search.


We also outline our approach to named-entity recognition and relation extraction -- 

, header extraction from standard-


The F$_{1}$ score (also F-measure) is the harmonic mean of precision ($p$) and recall $(r)$, as defined by the formula: F$_{1} = \frac{2rp}{r + p}$.
\end{description}	


This report presented our approach to building a conversational agent to sustain correspondence with advance fee fraud scammers over email.
We established the context of the problem and outlined our rationale in Section 1.2 and Chapter 2.

In this report we presented

This report presented our approach to building a conversational agent to sustain correspondence with advance fee fraud scammers over email.



This thesis presented our approach to fast low-rank metric learning. The need for
a low rank metric was motivated in the context of kNN (chapter 1 and section 2.1),
but we argue that learning a metric is useful whenever our algorithm relies on
dissimilarities. Our eorts were directed towards the already established method,
neighbourhood component analysis (NCA; section 3.1). We introduced a family
of algorithms to mitigate NCA's main drawback | the computational cost. In
our attempt of speeding up NCA, we encountered other interesting theoretical
and practical challenges. The answers to these issues represent an important part
of the thesis (sections 3.2 and 3.3). We summarize here our main contributions
and present the conclusions