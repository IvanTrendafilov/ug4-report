\chapter{Conclusions}
This report presented our approach to building a conversational agent for email correspondence with advance fee fraud scammers.
We established the context of the problem and outlined our rationale in Section 1.2 and Chapter 2. In Section 3.1 we provided an
overview of the architecture of the proposed system. In our implementation we encountered a number of interesting challenges.
We summarize here our main contributions.
\begin{description}
\item[Section 4.2] We introduced a set of collector modules which automate the process of obtaining new scam instances and fetching replies to existing conversation threads.
\item[Section 5.2] This section described key information extraction techniques. We detailed our approach to dealing with dirty data, header extraction from non-compliant emails, named-entity recognition, and relation extraction using bigrams and proximity search.
\item[Section 6.2] This section introduced the 419 corpus -- a corpus containing 32,000 advance fee fraud emails in 22 categories.
\item[Section 6.3] In this section we outlined our approach to building two Maximum Entropy classifiers. We used these classifiers to determine the variation of AFF for incoming messages and detect the presence or absence of personal questions in the body of a message.
\item[Section 7.2] We proposed a bucketing algorithm designed to group incoming messages into threads and maintain conversation state.
\item[Section 7.3] In this section we presented the cooperative and non-cooperative strategies used by our agent to keep the scammer engaged for as long as possible.
\item[Section 7.4] In this section we described our approach to response generation using classifiers, rules, hierarchical finite state machines and text snippets. For illustrative purposes, we showed an example of a generated message.
\end{description}	

\section{Future directions}
This project involved many ideas and interesting problems. Unfortunately, the time was limited and we could only explore some of the available paths. In this section we propose future directions for improving on our proposed prototype system.

For our prototype system, we implemented a natural language model which covers the \textit{Lottery, Orphans, Mystery shopper} advance fee fraud variations. This leaves 19 other unsupported classes. A future improvement, therefore, is to develop models to cover all remaining variations. In Section 8.3 we observed that \textit{Romance, Next of kin, Business, Loans} scams are most commonly encountered, therefore we recommend future work to start by supporting these first.

In Section 6.2 we introduced the 419 corpus. We believe these corpus can be useful for further research into advance fee fraud, as well as to develop more sophisticated statistical spam filters. As we discussed in Section 2.2, the current state of spam filters leaves room for improvement. Therefore, an interesting extension over this project is to explore whether it is feasible to train more effective spam filters, using the data in the 419 corpus.